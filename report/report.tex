\documentclass[9pt,oneside]{amsart}

\title{Symbolic repetability verification algorithm}
\author{
  Luca Zaninotto mat. 2057074 
  \and
  Davide Farinelli mat.2005758
  \and
  Matteo Budai mat. 2057217
}

\usepackage{xcolor}
\usepackage{listings}
\usepackage{hyperref}
\usepackage{multicol}
\usepackage{tikz}
\usepackage[a4paper,width=170mm,top=10mm,bottom=22mm,includeheadfoot]{geometry}

\hypersetup{
 pdfauthor={Luca Zaninotto},
 pdftitle={},
 pdfkeywords={},
 pdfsubject={Monitor to check the fairness of the railroad system},
 pdfcreator={Emacs 27.2 (Org mode 9.4.4)}, 
 pdflang={English}
}

\definecolor{codegreen}{rgb}{0,0.6,0}
\definecolor{codegray}{rgb}{0.5,0.5,0.5}
\definecolor{codepurple}{rgb}{0.58,0,0.82}
\definecolor{backcolour}{rgb}{1,1,1}

\lstdefinestyle{mystyle}{
    backgroundcolor=\color{backcolour},
    commentstyle=\color{codegreen},
    keywordstyle=\color{magenta},
    numberstyle=\tiny\color{codegray},
    stringstyle=\color{codepurple},
    basicstyle=\ttfamily\footnotesize,
    breakatwhitespace=false,
    breaklines=true,
    captionpos=b,
    keepspaces=true,
    numbers=left,
    numbersep=5pt,
    showspaces=false,
    showstringspaces=false,
    showtabs=false,
    tabsize=2
}

\lstset{style=mystyle}

\begin{document}
\begin{abstract}
  Algorithm for invariant verification in a finite state model through
  symbolic representation of regions, using pynusmv to encode and
  model the system itself
\end{abstract}
\maketitle
\setlength{\columnsep}{20pt}
\begin{multicols}{2}
  \section{The algorithm}\label{algo}
  The algorithm we developed works in two parts. Given a set of
  states, a transition system and a temporal logic invariant in the
  shape of
  \[
    \square \diamondsuit f \rightarrow \square \diamondsuit g
  \]
  first it discovers weather the LTL formula is respected or not,
  then, if the LTL formula is not respected provides a
  counterexample. The first step in order to do this is to negate the
  formula, resulting in the logic formula
  \[
    \square \diamondsuit f \rightarrow \diamondsuit \square \neg g
  \]

  The algorithm is divided into two functions, one that checks weather
  there's a cycle that repeats $f$ and $\neg g$, the other, that
  builds the cycle if there's one
  \section{region exploration}\label{explore}
  To verify weather there's a cycle that respects our condition to
  invalidate the formula is the following
  \begin{lstlisting}[language=Python]
  	def research(fsm, f, g):
  		reach = fsm.init
  		new = fsm.init
  		while fsm.count_states(new) > 0:
  			new = fsm.post(new) - reach
  			reach = reach + new
  		recur = reach & f & (~g)
  		while fsm.count_states(recur) > 0:
	  		new = fsm.pre(recur) & (~g)
		  	reach = new
		  	while fsm.count_states(new) > 0:
		  		reach = reach + new
		  		if recur.entailed(reach):
		  			return False, gen_counterex(fsm, f, g, reach)
		  		new = (fsm.pre(new) - reach) & (~g)
			recur = recur & reach
	  	return True, None
  \end{lstlisting}\label{code:reachable}
  The algorithm is basically a symbolic nested search algorithm to
  find a subset of nodes containing a cycle that invalidates our
  original formula (or better, respects the negation of it).

  \section{Generating a counterexample}\label{back}

  Next, building on the result of the research function, we can build
  the counterexample for the considered transition system. We do so by
  reversing the original list of states in the reach subset of our
  system, since to search for the cycle we needed to use the pre-image,
  instead of the post-image of the considered nodes, exploring the
  state in a backwards flow.
  \begin{lstlisting}[language=Python]
      def gen_counterex(fsm, reach):
      	states = []
      	while fsm.count_states(reach):
      		state = fsm.pick_one_state(reach)
      		states.append(state)
      		reach = reach - state
      	last_state = states[-1]
      	states.reverse()
      	states.append(last_state)
      	counterex = ()
      	for (s1, s2) in zip(states, states[1:]):
      		inputs = fsm.get_inputs_between_states(s1, s2)
      		if inputs != pynusmv.dd.BDD.false():
      			inputt = fsm.pick_one_inputs(inputs)
      			counterex += (s1.get_str_values(), inputt.get_str_values())
      		else:
      			counterex += (s1.get_str_values(), {})
      	counterex += (states[-1].get_str_values(), )
      	return counterex
  \end{lstlisting}
  The counterexample is generated and returned to the main function
  that displays it.
\end{multicols}
\end{document}
